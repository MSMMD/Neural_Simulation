\documentclass[12pt]{article}
\usepackage{amsmath, amsthm, amssymb}
\usepackage[colorlinks=true, urlcolor=blue, linkcolor=cyan]{hyperref}
\usepackage[shortlabels]{enumitem}
\usepackage{silence}
\usepackage{tikz}
\usepackage[margin=20mm]{geometry}

\title{Reuniões - Iniciação científica}
\author{Mateus Scheffer Mendes Maziviero Dolce}
\date{}

\begin{document}

\maketitle

%\tableofcontents
%\pagebreak
\section{\date{19/09/25}}
    \subsection{}
        \begin{itemize}
            \item \textbf{Sinapses químicos:} Seja $w_{i\rightarrow j}$ o peso de quanto $i$ estimula $j$ após disparo do neurônio $i$.
            \item \textbf{Reset do potencial:} Sempre que um neurônio dispara, seu potencial retorna para o repouso.
            \item \textbf{Potencial e proporcional à probabilidade de disparo:} Quanto maior o potencial do neurônio $i$, mais provável ocorrer um dispara nele.
            \item \textbf{Canais de vazamento:} Quando livre de estimulos, os potenciais tendem naturalmente para o repouso.
        \end{itemize}

    \subsection{}
        \paragraph{}Sendo $(t_j)_k$ o tempo do último disparo de $j$ após o momento $k$, e $t$ o momento atual.
        \begin{equation*}
            \begin{aligned}
                V_{t+1}(j) &= \text{Potencial de }j\text{ no tempo }t\\
                        &= \sum_{s = (t_j)_k}^{t+1} (\sum_{i \neq j} w_{i\rightarrow j} X_s(i) \rho^{(t+1)-s} )\\
                        &= \sum_{s = (t_j)_k}^{t} (\sum_{i \neq j} w_{i\rightarrow j} X_s(i) \rho^{(t+1)-s} ) + \sum_{i \neq j} w_{i\rightarrow j} X_{t+1}(i) \rho^{0}\\
                        &= \rho \sum_{s = (t_j)_k}^{t} (\sum_{i \neq j} w_{i\rightarrow j} X_s(i) \rho^{(t)-s} ) + \sum_{i \neq j} w_{i\rightarrow j} X_{t+1}(i) \rho^{0}\\
                        &= \rho V_{t}(j) + \sum_{i \neq j} w_{i\rightarrow j} X_{t+1}(i)
            \end{aligned}
        \end{equation*}

    \subsection{}
        \paragraph{}Leitura do capítulo 2 do livro \textbf{"Probabilistic Spiking Neuronal Nets,
        Antonio Galves · Eva Löcherbach Christophe Pouzat"}.
\newpage

\section{\date{26/09/25}}

\paragraph{}Realizar simulação manual com:
\begin{itemize}
    \item $N=5$
    \item $W_{i\rightarrow j} = 1 \text{   } \forall i\neq j \text{ com } i, j \in I$
    \item $V_{0}(i) = 1 \text{ } \forall \text{ } i \in I$
    \item $n=5$
    \item $\rho=1$
    \item $\varphi(x)= min(\frac{x}{2}, 1)$
\end{itemize}

\section{\date{03/10/25}}

\paragraph{}Definir uma função $\varphi$ para que $\varphi( \sum_{j\in I} w_{j\rightarrow i})$ não seja
"muito perto" de $1$, para todo $i \in I$.
\paragraph{}Para fazer isso, analisaremos o potencial médio. Isto é, $\bar{V_t} = \frac{1}{N}\sum_{i \in I} V_t(i)$.
\paragraph{}Realizar simulação com pesos que podem ser negativos.

\subsection{Análises}
\paragraph{}Padronizarei os pesos para que fiquem no intervalo $[-1, 1]$, para isso, dividirei todos os pesos pelo maior
valor absoluto dentre eles.\\
Então, a maior variação que pode acontecer será de tamanho $(N-1)$.

\section{\date{17/10/25}}

\paragraph{}É mais provável que os neurônio estimulem uns aos outros. Então, tomaremos que $50\%$ dos pesos estimulam e $30\%$ inibe.

\section{\date{31/10/25}}

\paragraph{}Sendo $\epsilon = \varphi(0)$, $p_{w^+}$ e $p_{w^-}$ os pesos esperados positivo e negativo das arestas,
respectivamente e o $p_\mathbf{e}$ o valor que desejamos do $\varphi(\bar V)$. Definiremos $\varphi$ como
\begin{gather*}
    \varphi = \frac{1}{1+ e^{-(x/a - b)}}\\
    b = \ln(1/\epsilon - 1)\\
    a = \frac{(p_{w^+} - p_{w^-})(N-1)}{p_\mathbf{e}/\epsilon}
\end{gather*}

\newpage

\section{\date{05/11/25}}

\subsection{Como construir o relatório}
\begin{itemize}
    \item Introdução/motivação.
    \item Definição teórica do modelo.
    \item Tunning da $\varphi$.
    \item Validação.
\end{itemize}

\section{\date{14/11/25}}
    \begin{itemize}
        \item Atualizações no CV lattes
        \item Realizar validação do $\varphi$ (Kolmogorov Smirnof)
    \end{itemize}
            
\end{document}